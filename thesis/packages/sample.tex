% Scanning electron microscopy
% Author: Eric Jensen
\documentclass[a4paper,10pt]{article}
\usepackage{tikz}
\usetikzlibrary{%
	calc,%
	decorations.pathmorphing,%
	fadings,%
	shadings%
}
\renewcommand*{\familydefault}{\sfdefault}

\begin{document}
\begin{tikzpicture}
  \draw[gray,fill=gray,path fading=south] (0.5,-2) rectangle +(5,-2);% sample
    \begin{scope}[decoration={snake,amplitude=.5mm,
        segment length=2mm,post length=1mm}]
     \draw[decorate,blue,->] (1,-.5) to (2.0,-2);% auger	
	  \draw[decorate,blue,->] (7,-1) to (8,-1);% auger
      \draw[decorate, decoration={snake, segment length=1.5mm, amplitude=.7mm,post length=1mm},red,->] (2,-2) to  [bend right](4.5,-2);% auger
	   \draw[decorate, decoration={snake, segment length=1.5mm, amplitude=.7mm,post length=1mm},red,->] (7,-1.5) to (8,-1.5);% auger
      \draw[decorate,decoration={snake, segment length=2.5mm, amplitude=.6mm,post length=1mm},orange,->] (4.5,-2) to (6,0);% cathodlimuescence
	  \draw[decorate,decoration={snake, segment length=2.5mm, amplitude=.6mm,post length=1mm},orange,->] (7,-2) to (8,-2);% cathodlimuescence
	   \draw[decorate,decoration={snake, segment length=2.5mm, amplitude=.6mm,post length=1mm},green,->] (4.5,-2) to (6,-3);% cathodlimuescence
	   \draw[decorate,decoration={snake, segment length=2.5mm, amplitude=.6mm,post length=1mm},green,->] (7,-2.5) to (8,-2.5);% cathodlimuescence
	  \begin{tikzpicture}[scale=0.1,x=0.5cm,y=0.5cm,line width=4pt,draw=yellow]
\draw (1,0) node[draw,circle,minimum size=5cm,scale=0.08] (TheSun) {};
    \foreach \angle in { 0,45,...,359  }
    {
        \draw [rotate around={\angle:(TheSun.center)}]
            ($(TheSun.center) + (5.5,0)$)
                   -- +(0,-1.5)
                   -- +(3,0)
                   -- +(0,1.5)
                   -- cycle;
     }

\end{tikzpicture}
    \end{scope}


		
  %labels
  \draw (0.5,-4) node[above right] {\footnotesize Material};
  \draw (8.1,-1.25) node[above right] {\footnotesize -incoming light};
  \draw (8.1,-1.75) node[above right] {\footnotesize -penetrated light};
   \draw (8.1,-2.25) node[above right] {\footnotesize -reflected light};
    \draw (8.1,-2.75) node[above right] {\footnotesize -transmitted light};
   \draw (2,-2) node[above right] {\footnotesize $(t_{i},x_{i},\lambda_{i})$};
    \draw (4.85,-2) node[above right] {\footnotesize $(t_{0},x_{0},\lambda_{0})$};
   \draw (0.5,-4) node[above right] {\footnotesize Material};
  \draw (3.5,0) ++(145:2) ++(255:0.3) ++(125:0.8) node[left]
    {\footnotesize Light Source};
\end{tikzpicture}
\end{document}