
\section{Surface Light Field}
\label{section:slf}
If we take BTF and assume

\begin{itemize}
\item fixed illumination direction $\theta_{i} = const$, $\phi_{i}=const$
\end{itemize}
we will get a 4-D Surface Light Field model (SLF)
 \begin{center}
$SLF(x,\theta_{o} ,\phi_{o})$.
 \end{center}
 
 SLF model is a simple a textural model and is a subset of BTF. 
 SLF is used when illumination direction is not varying in the rendering scene, but the view direction varies.
 Thus, such model is favoured for its greater computational efficiency for such cases.

\section{Surface Reflectance Field}
\label{section:srf}
Analogously as for SLF, if we take BTF and assume

\begin{itemize}
\item fixed view direction $\theta_{o} = const$, $\phi_{o}=const$
\end{itemize}
we will get a 4-D Surface Reflectance Field model (SRF)
 \begin{center}
$SRF(x,\theta_{i} ,\phi_{i})$.
 \end{center}
 

 SRF is another very popular variant of image-based rendering and is also a subset of BTF. 
 In this case many images are taken under varying light directions with fixed view point\cite{star2004}. 
 For instance, Debevec et al. \cite{debevec} recorded the appearance of human face while a
light source was rotating around the face. Rendering the human face realistically is always a struggle in computer graphics.
Due to complex reflectance characteristics of the human face, common texture mapping usually fails under varying illumination.
 Debevec et al. acquired approximately two thousand images under different light positions and could render the face under arbitrary lighting for original point views. 
 Also, in the second part of the paper Debevec et al. present a technique to extrapolate a complete reflectance field from the acquired data, i.e. extrapolation allows them to render the face from novel viewpoints.


\section{2-D Texture/Bump-Maps}
\label{section:2dtexture}
By getting rid of the dependence both on light and viewing directions, we are left with a 2D texture map (TM) or bump map (BM)
 \begin{center}
$TM(x)$,
 \end{center}
 which represents spatial variations of surface color or surface
normal orientation, respectively\cite{dong}. 
These 2D functions do not explicitly represent light scattering, but instead provide appearance attributes that are typically used in conjunction with BRDFs in rendering \cite{dong}. 
As for BMs, they consist of an array of normals that are used instead of the normals of the given geometry and can be used to simulate a bumpy surface.




\subsection{Advantages}
\label{section:Advantages}


\begin{tabular}{ |l|l| }
  \hline
  \multicolumn{2}{|c|}{Advantages} \\
  \hline
  GRF &  most descriptive and physically correct representation  \\
  BSSRDF & the best approximation of GRF \\
  BTF & currently the best approximation of GRF in practice \\
  BSSDF & best for homogeneous materials for which subsurface scattering is very important feature \\
  BRDF &  an optimal representation with many developed models \\
  SVBRDF & measurement, compression, and modelling are simpler than for the BTF \\
  SLF & a simplified function, which accounts for view direction variance (BTF subset) \\
  SRF & a simplified function, which accounts for light direction variance (BTF subset) \\
  TM/BM & the simplest visual texture function, widely used in a computer graphics\\
  \hline
\end{tabular}


\subsection{Disadvantages}
\label{section:Disadvantages}


\begin{tabular}{ |l|l| }
  \hline
  \multicolumn{2}{|c|}{Disadvantages} \\
  \hline
  GRF &  so far infeasible to measure, model\\
  BSSRDF &  difficulties in reliable measurement and in modeling \\
  BTF &  expensive measurement, demanding on computing resources and mathematical tools \\
  BSSDF & represent only homogeneous materials\\
  BRDF & suits best for nearly flat, opaque materials, breaks physical laws \\
  SVBRDF & worse visual quality than BTF and suits for nearly flat, opaque materials \\
  SLF & illumination variation is not present \\
  SRF & viewing variation is not present  \\
  TM/BM & a sketchy textural approximation, which ignores most important appearance features\\
  \hline
\end{tabular}
