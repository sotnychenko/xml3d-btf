\chapter{Interpolation of Unmeasured Directions}
\label{chapter:interpolation}


BTF datasets are measured for discrete number of angles, thus we would need to perform interpolation for unknown not measured angels.
We will employ barycentric coordinates interpolation. 
However, it is very computational heavy, so the following approximation algorithm proposed by Hatka and Haindl \cite{btfblender} will be used.

Assume that we have found triangle $P_{1}P_{2}P_{3}$ which bounds our point P, which we want to interpolate. Figure \ref{fig:acquisition_example} demonstrates hemisphere on which triangle $P_{1}P_{2}P_{3}$ lies.
$Y_{P}$ denotes desired pixel color. 
So, generally speaking linearly interpolation of that pixel will be $Y_{P}=w_{1}Y_{P1} + w_{2}Y_{P2} + w_{1}Y_{P2}$, 
where $Y_{P1},Y_{P2},Y_{P3}$ correspond to measured pixel color of positions $P_{1},P_{2},P_{3}$ accordingly. Weights $w_{1},w_{2},w_{3}$ are normalized and sum up to $1$.

Weights defined as volumes $V_{1},V_{2},V_{3}$ which correspond to $PP_{2}P_{3}O$, $PP_{3}P_{1}O$, $PP_{1}P_{2}O$ tetrahedrons, where $O=(0,0,0)$.
All volumes are normalized, which means $V_{i}=\frac{V_{i}}{\sum_{i=1}^{3}V_{i}}$. Volumes calculated as determinates of $4\times 4$ vectors

{\centering $V_{1}=\frac{1}{6}\left | det(PP_{2}P_{3}O) \right |$ \\}
{\centering $V_{2}=\frac{1}{6}\left | det(PP_{3}P_{1}O) \right |$ \\}
{\centering $V_{3}=\frac{1}{6}\left | det(PP_{1}P_{2}O) \right |$ \\}