\chapter{Conclusions and Future Work}
\label{chapter:conclusions}
 Bidirectional Texture Functions are currently the best texture representation for the materials 
 that contain high frequencies both in the angular and spatial domain \cite{mueller-2003-compression}.
 Thousands of images has to be taken to sample the appearance of such materials.
 Due to huge sizes of acquired BTFs real-time rendering is impossible to achieve without suitable compression.
 Thus, BTFs are still rarely used and staying in a state of the art in computer graphics. 

\section{Summary}

 In this work we achieved the realistic and real-time rendering for a scene with several BTFs.
We showed that real-time rendering is possible even on the mobile devices.
With constantly improving hardware BTFs have a bright future in computer graphics.

 Considering that we render BTFs in a browser, we managed to reduce the latency that is caused by the data transmission, i.e.
 by streaming  principal components individually.
 We managed to improve the decompression error that is caused by floating point imprecisions.
 The scaling of the compressed BTF data before converting it into the textures decreased the decompression error approximately by $5\%$.
 As a result this also improved the real-time frame rate and the compression ratio due to reduction in the number of used components.
 Last but not least, our method produces better decompression errors compared to other implemented PCA methods \cite{haindl}.


Finally, our method is flexible and allows for balancing between the visual quality, memory usage and computational effort.
\section{Future work}
\label{section:future_work}

Based on our results several directions for the future work can be made.
First of all, multiple streaming of BTFs can be implemented.
Secondly, several optimizations of the BTF shader performance can be made.
For instance, it is possible to reduce some of the calculations such as computation of the interpolation weights. 
They can be precomputed and stored in a cube-map \cite{haindl}.
Other calculations such as computation of the bidirectional normals can be precomputed and stored along with a 3D mesh.

Last but not least, there is a room for further compression.
 For instance, it can be achieved by compressing the PCA parameters with wavelet compression \cite{webglbtfstreaming} or with entropy encoding \cite{gpu_gems}. 
However, the decompression cost will grow, but it can be worth if further memory savings are necessary. 