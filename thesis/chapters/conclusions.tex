\chapter{Conclusions and Future Work}
\label{chapter:conclusions}
Recently Bidirectional Texture Functions becoming more and more popular for the realistic rendering.
Until now lots of research has been done in this field. 
However, BTFs are still rarely used in the graphical applications, 
due to  sophisticated measurement of BTFs and the large size of the acquired data.
Thus, BTFs are still staying as the state of the art in the computer graphics. 


\section{Summary}
In this thesis we showed that the Bidirectional Texture Function is a powerful texture representation and 
that the real-time frame rate can be achieved during rendering.
We looked into different compression methods available for BTFs and decided to work with PCA.
We came up with a flexible PCA compression method based on previous works.
We perform PCA on the chosen number of neighbour camera directions.
This allows for achieving a compression ratio up to $1:100$, while having a small decompression error by adapting the number of principal components.
Last but not least, by using small number of principal components it is possible to achieve real-time frame rates.

We have built a WebGL framework for rendering 3D objects realistically using Bidirectional Texture Functions.
To reduce the delay caused by the BTF data transmission from the server to the client, we use WebSockets streaming.
By streaming the principal components we are able to show intermediate image results of the textured 3D object.
The first previews of the rendering are immediately available for the user.


\section{Future work}
\label{section:future_work}
One the possible future works can be done streaming multiple BTFs for the scene.
Improving the rederning performnace by reducing the number of instructions in the GPU.
For instance, it is possible to precompute interpolations coeficients (cite) and store them in cubemap.
Also, improve the compression further by using vector quantization technique cite gpu gems.
Also, it is possible to stream the meshes and store them on the web server.
