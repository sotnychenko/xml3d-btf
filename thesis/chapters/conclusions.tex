\chapter{Conclusions and Future Work}
\label{chapter:conclusions}
 Bidirectional Texture Functions are currently the best texture representation for materials 
 that contain high frequencies in both the angular and spatial domain \cite{mueller-2003-compression}.
 Thousands of images have to be taken to sample the appearance of such materials.
 Due to the huge size of an acquired BTF real-time rendering is impossible to achieve without suitable compression.
 Thus, BTFs are still rarely used and staying in a state of the art in computer graphics. 

\section{Conclusions}

 In this work we achieved the realistic and real-time rendering for a scene with several BTFs.
We showed that real-time rendering is possible even on mobile devices.
With constantly improving hardware BTFs have a bright future in computer graphics.

 Considering that we render BTFs in a browser, we managed to reduce the latency that is caused by the data transmission, i.e.
 by streaming  principal components individually.
 We managed to improve the decompression error that is caused by floating point imprecisions.
 The scaling of the compressed BTF data before converting it into the textures decreased the decompression error approximately by $5\%$.
 As a result this also improved the real-time frame rate and the compression ratio due to reduction in the number of used components.
 Last but not least, our method produces better decompression errors compared to other implemented PCA methods \cite{haindl}.


Finally, our method is flexible and allows for balancing between the visual quality, memory usage and computational effort.
\section{Future Work}
\label{section:future_work}

Based on our results several directions for future work are possible.
First of all, several optimizations of the BTF�s performance.
For instance, to reduce some of the calculations such as computation of the interpolation weights, a precomputed cube-map can be used \cite{haindl}.
Other calculations such as computation of the binormals can be precomputed and stored along with a 3D mesh.

Furthermore, there is room for further compression.
 For instance, PCA parameters can be further compressed using wavelet compression \cite{webglbtfstreaming} or entropy encoding \cite{gpu_gems}, if further memory savings are necessary. 