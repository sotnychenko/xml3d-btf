\chapter{BTF data representations}
\label{chapter:representations}




Before doing any compression step it is important to chose a suitable data representation for BTF data.
Suitable presentation can enormously influence the final result of BTF rendering.
 Basically, there are two common arrangements for BTF, i.e. as a set of original rectified and registered textures (\emph{image-wise} representation) and a set of ABRDFs (\emph{pixel-wise} representation).
 ABRDF was defined in chapter \ref{section:brdf}. In this case it is called \emph{apparent},
 because BTF includes effects such self-occlusions, sub-surface scattering and other complex effects which violate two basic properties of BRDF.
Both representations can be mathematically expressed as

{\centering $\,\,\,\,\,\,\,BTF_{Texture}=\left \{I_{(\theta_{i} ,\phi_{i},\theta_{o} ,\phi_{o}) }  \mid  (\theta_{i},\phi_{i},\theta_{o} ,\phi_{o})\in M \right \}$\\}
{\centering $\,\,\,\,\,\,\,\,\,\,\,\,\,\,\,\,BTF_{ABRDF}=\left \{P_{(x) } \,\,\,\,\,\,\,\,\,\,\,\,\,\,\,\,\,\,\mid  (x)\in I_{(\theta_{i} ,\phi_{i},\theta_{o} ,\phi_{o})}\subset \mathbb{N}^{2}\right \}$ \\}


where $M$ denotes a set of images $I_{(\theta_{i} ,\phi_{i},\theta_{o} ,\phi_{o})}$ measured for different light and camera directions $(i,o)$ accordingly.
$BTF_{ABRDF}$ denotes a set of $P_{(x)}$ images, where each of them stores light intensity for a fixed spatial position $x$ for all possible light and camera directions, 
i.e. a domain for a $P_{(x)}$ image is $(n_{i}\times n_{o})\subset \mathbb{N}^{2}$, where $n_{i}$ and $n_{o}$ are number of light and camera directions accordingly.
In our case with Bonn datasets the images are given with $256\times256$ resolution for $81\times81$ directions, 
then the size of one single image for both representations are $\left | I \right | = 3\times256\times256$ and $\left | P \right | = 3\times81\times81$, where $3$ stands for RGB channels.

Basically, one representation, i.e. $BTF_{Texture}$ is used for compression methods that do analysis on the whole sample plane, 
while $BTF_{ABRDF}$ representation allows better pixel-to-pixel comparisons, which can give a big advantage for methods that employ pixel-wise compression, e.g. BRDF based models.
Also, such arrangement provides images with lower variance \cite{haindl}, which can allow better compression results in certain scenarios.
But, anyway both representations posses the same information, and any compression method can use either of them.

 
M{\"u}ller et. al. \cite{mueller-2003-compression, haindl} claim that ABRDF representation works slightly better in comparison to image-wise representation. 
The advantage of ABRDF is that a compression ratio can be 10 times better and reconstruction error is slightly smaller than the image-wise representation.
Also, Borshukov et. al. \cite[Ch.\ 15]{gpu_gems} chose ABRDF representation, claiming that it provides better compression ratio.
The reason why ABRDF provides better compression ratio is because it provides better pixel-to-pixel comparison than image-wise. 
Each variable vector of ABRDF arrangement depends only on a surface complexity \cite{mueller-2003-compression}, i.e. 
on a variation of reflection properties at a spatial point of the surface.
On the contrary, image-wise variable depends on the whole measured image plane, thus it is logical that it may provide lower compression rate due to stronger variations.
After all, we have chosen to employ ABRDF data representation based on above reasons, which allowed us to achieve $1:100$ compression ratio. 
