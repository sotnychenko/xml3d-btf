


\documentclass[border=2pt,tikz]{standalone}
\usepackage{tikz-3dplot}
% Define a point.
% #1 = name of the point
% (#2,#3,#4) is the location.
% Also creates a coordinate node of name #1 at the location.
\newcommand\definePointByXYZ[4]{
    \coordinate (#1) at (#2,#3,#4);
    \expandafter\gdef\csname tsx@point@#1\endcsname{
        \def\tsx@point@x{#2}
        \def\tsx@point@y{#3}
        \def\tsx@point@z{#4}
    }
}
\begin{document}
\pgfmathsetmacro{\alpha}{55}
\pgfmathsetmacro{\beta}{60}
\tdplotsetmaincoords{\alpha}{\beta} % Perspective on the main coordinate system
\pgfmathsetmacro{\radius}{0.8} % radius of the circle

\begin{tikzpicture}[scale=5,tdplot_main_coords]
% Draw circle in the un-rotated coordinates
% \draw[blue,tdplot_screen_coords] (0,0,0) circle (\radius);

% draw coordinate vectors for reference
\draw[->] (-1,0,0) -- (1,0,0) node[anchor=north east]{$x$};
\draw[->] (0,-1,0) -- (0,1,0) node[anchor=north west]{$y$};
\draw[->] (0,0,-1) -- (0,0,1) node[anchor=south]{$z$};

% draw the "visible" and  "hidden" portions of the circumference as a solid and dashed semi-circles, parametrically 
\draw[red,dashed,domain={-180+\beta}:\beta] plot ({\radius*cos(\x)}, {\radius*sin(\x)});
\draw[red,dashed,domain=\beta:{180+\beta}] plot ({\radius*cos(\x)}, {\radius*sin(\x)});

\pgfmathsetmacro{\radius}{.75} % radius of the circle
\draw[red,dashed,domain={-180+\beta}:\beta] plot ({\radius*cos(\x)}, {\radius*sin(\x)},0.1);
\draw[red,dashed,domain=\beta:{180+\beta}] plot ({\radius*cos(\x)}, {\radius*sin(\x)},0.1);

\pgfmathsetmacro{\radius}{.65} % radius of the circle
\draw[red,dashed,domain={-180+\beta}:\beta] plot ({\radius*cos(\x)}, {\radius*sin(\x)},0.2);
\draw[red,dashed,domain=\beta:{180+\beta}] plot ({\radius*cos(\x)}, {\radius*sin(\x)},0.2);
\pgfmathsetmacro{\radius}{.55} % radius of the circle
\draw[red,dashed,domain={-180+\beta}:\beta] plot ({\radius*cos(\x)}, {\radius*sin(\x)},0.3);
\draw[red,dashed,domain=\beta:{180+\beta}] plot ({\radius*cos(\x)}, {\radius*sin(\x)},0.3);
\pgfmathsetmacro{\radius}{.45} % radius of the circle
\draw[red,dashed,domain={-180+\beta}:\beta] plot ({\radius*cos(\x)}, {\radius*sin(\x)},0.35);
\draw[red,dashed,domain=\beta:{180+\beta}] plot ({\radius*cos(\x)}, {\radius*sin(\x)},0.35);
\pgfmathsetmacro{\radius}{.3} % radius of the circle
\draw[red,dashed,domain={-180+\beta}:\beta] plot ({\radius*cos(\x)}, {\radius*sin(\x)},0.4);
\draw[red,dashed,domain=\beta:{180+\beta}] plot ({\radius*cos(\x)}, {\radius*sin(\x)},0.4);
\pgfmathsetmacro{\radius}{.2} % radius of the circle
\draw[red,dashed,domain={-180+\beta}:\beta] plot ({\radius*cos(\x)}, {\radius*sin(\x)},0.45);
\draw[red,dashed,domain=\beta:{180+\beta}] plot ({\radius*cos(\x)}, {\radius*sin(\x)},0.45);


\filldraw[fill=blue,fill opacity=0.6] (0.2,-0.2,0) -- (0.2,0.2,0) --  (-0.2,0.2,0) -- (-0.2,-0.2,0) -- cycle ;


  \draw[thin,<-] (0.21,0.21,0) -- (0.8,0.8,0.1) node[right] {Sampled Surface};
  
% \draw[thin,<-] (0.05,0.05,0.9) -- (0.7,0.7,0.95) node[right] {Camera Position};
 
 \draw[thin,<-] (.2,.35,0.65) -- (0.8,0.85,.65) node[right] {Camera and Light trajectories};
  
%\definePointByXYZ{mypoint}{0}{0}{0.9};
 %   \draw (mypoint) circle [radius=0.7pt];
%	\fill (mypoint) circle [radius=0.7pt];
	


	
%	\fill[ball color=yellow] (-.2,.93) circle (0.051 cm);
	
	
	

\end{tikzpicture}

\end{document}